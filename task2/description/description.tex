\documentclass[a4paper,11pt]{article}
\usepackage[utf8]{inputenc}
\usepackage[czech]{babel}
\usepackage{graphicx}
\usepackage{listings}
\usepackage[unicode]{hyperref}
\usepackage{url}
\usepackage{hyperref}
\usepackage{mathtools}


\title{\textbf{MI-PAP}}
\author{Dominik Soukup, Jiří Kadlec}
\date{01. 03. 2017}
\begin{document}
\maketitle
\thispagestyle{empty}
\newpage
\tableofcontents
\thispagestyle{empty}
\newpage


\section{Implementace sekvenčního algoritmu}
%\subsection{Definice problému}

\subsection{Popis Dijkstrova algoritmu}
Algoritmus je možné chápat jako zobecněné prohledávání do šířky. Po provedení nám dává řešení nejkratších
cest z jednoho počátečního uzlu do všech ostatních. Předpokladem algoritmu je, že žádná hrana není záporně
ohodnocena. V této implementaci jsou navíc všechny hrany ohodnoceny kladně.

Pro složitost vytvořeného algoritmu platí:
Počáteční inicializace použitých struktur \textit{O(${nodes}$)}. Délka hlavního cyklu závisí na počtu uzlů.
V rámci tohoto cyklu se vybírá následující uzel a zpracují se jeho potomci. Pro výběr uzlů byla použita
prioritní fronta. Celková složitost je tedy  \textit{O(${nodes . log(nodes)}$)}.
Následně se zpracovávají všechny potomci zvoleného uzlu. Při změně ceny je potřeba vložit uzel do prioritní
fronty. Složitost této časti je \textit{O(${edges . log(nodes)}$)}. Nakonec je nutné počítat s tím, 
že se výpočet spouští z každého uzlu. 
Celková složitost implementovaného Dijsktova algoritmu je\-

\textit{O(${nodes*(nodes+nodes . log(nodes)+edges . log(nodes))}$)}

\subsection{Popis Floyd-Warshallova algoritmu}
Výsledkem tohto algoritmu jsou nejkratší cesty mezi všemi páry uzlů.

Pro složitost vytvořeného algoritmu platí:
Počáteční inicializace použitých struktur \textit{O(${nodes}$)}. Následně se provádějí 3 vnořené cykly.
Celková složitost je tedy \textit{O(${nodes^3}$)}


\end{document}