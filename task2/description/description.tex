\documentclass[a4paper,11pt]{article}
\usepackage[utf8]{inputenc}
\usepackage[czech]{babel}
\usepackage{graphicx}
\usepackage{listings}
\usepackage[unicode]{hyperref}
\usepackage{url}
\usepackage{hyperref}

\title{\textbf{MI-PAP}}

\author{Dominik Soukup, Jiří Kadlec}
\date{01. 03. 2017}
\begin{document}
\maketitle
\thispagestyle{empty}
\newpage
\tableofcontents
\thispagestyle{empty}
\newpage

\section{Implementace sekvenčního algoritmu}
%\subsection{Definice problému}

\subsection{Popis Dijkstrova algoritmu}
Algoritmus je možné chápat jako zobecněné prohledávání do šířky. Po provedení nám dává řešení nejkratších
cest z jednoho počátečního uzlu do všech ostatních. Předpokladem algoritmu je, že žádná hrana není záporně
ohodnocena. V této implementaci jsou navíc všechny hrany ohodnoceny kladně.

Pro složitost vytvořeného algoritmu platí:
Počáteční inicializace použitých struktur \textit{O(${nodes}$)}. Délka hlavního cyklu závisí na počtu uzlů.
V rámci tohoto cyklu se vybírá následující uzel a zpracují se jeho potomci. Pro výběr uzlů byla použita
prioritní fronta. Celková složitost je tedy  \textit{O(${nodes . log(nodes)}$)}.
Následně se zpracovávají všechny potomci zvoleného uzlu. Při změně ceny je potřeba vložit uzel do prioritní
fronty. Složitost této časti je \textit{O(${edges . log(nodes)}$)}. Nakonec je nutné počítat s tím, 
že se výpočet spouští z každého uzlu. 
Celková složitost implementovaného Dijsktova algoritmu je\-

\textit{O(${nodes*(nodes+nodes . log(nodes)+edges . log(nodes))}$)}

\subsection{Popis Floyd-Warshallova algoritmu}
Výsledkem tohto algoritmu jsou nejkratší cesty mezi všemi páry uzlů.

Pro složitost vytvořeného algoritmu platí:
Počáteční inicializace použitých struktur \textit{O(${nodes}$)}. Následně se provádějí 3 vnořené cykly.
Celková složitost je tedy \textit{O(${nodes^3}$)}

Floyd-Warshallův algoritmus díky třem vnořeným cyklům a jednoduché podmínce je implementačně jednoduchý a pseudokod pro nalezení nejkratších cest vypadá násedovně:

\lstset {language=C++}
\begin{lstlisting}
for k in 1 to n do
 for i in 1 to n do
   for j in 1 to n do
      if D[i][j] > D[i][k] + D[k][j] then
         D[i][j] = D[i][k] + D[k][j]
         P[i][j] = P[k][j]
\end{lstlisting}

kde matice \textbf{D} je matice délek a matice \textbf{P} je matice předchůdců.

\section{OpenMP pro x86 a Xeon Phi)}
\subsection{Dijk}

\subsection{ Floyd-Warshall}
Implementace floyd-warshalova algoritmu je v podstatě jednoduchá, ale tělo nejvnitřnějšího for cyklu se dá přepsat několika způsoby.
Na základě toho jsme zkoušeli, který způsob běží nejrychleji a jestli se nám podařilo vektorizovat tento algoritmus. 


Jednotlivé implementace a jejich zhodnocení si popíšeme v nasledujících podkapitolách.

\subsubsection{Klasický Floyd-Warshall}


Implementace vypadá následovně:

\lstset {language=C++}
\begin{lstlisting}
for ( int k = 0; k < n; ++k)
  for ( int i = 0; i < n; ++i) 
    for ( int j = 0; j < n; ++j) {
      if (D[i][j] >  D[i][k] + D[k][j]) {
        D[i][j] = D[i][k] + D[k][j];
        P[i][j] = P[k][j];
    }
\end{lstlisting}

Toto řešení není vektorizované a následné časy pro x86 a Xeon Phi jsou následovné.


\begin{tabular}{|c|c|c|c|}
	\hline vlákna\textbackslash  uzlů & 1000 & 1500 & 2000 \\ 
	\hline 1 & 1.11602 & 3.79188 & 9.18048 \\ 
	\hline 2 & 0.564859 & 1.84135 & 4.34206 \\ 
	\hline 4 & 0.288087 & 0.972909 & 2.17533 \\ 
	\hline 6 & 0.196072 & 0.670776 & 1.49812 \\ 
	\hline 8 & 0.15429 & 0.479784 & 1.11342 \\ 
	\hline 10 & 0.130064 & 0.38863 & 0.886158 \\ 
	\hline 12 & 0.185516 & 0.514168 & 0.932699 \\ 
	\hline 24 & 0.140569 & 0.398891 & 0.76749 \\ 
	\hline 
\end{tabular} 

\begin{tabular}{|c|c|c|c|}
	\hline vlákna\textbackslash & 1000 & 1500 & 2000 \\ 
	\hline 61 & 0.937226 & 1.11005 & 2.49671 \\ 
	\hline 122 & 0.662043 & 0.9634 & 1.1.9717 \\ 
	\hline 183 & 0.457726 & 0.973209 & 1.90224 \\ 
	\hline 244 & 0.485168 & 0.886052 & 1.70336 \\ 
	\hline 
\end{tabular} 

\subsubsection{Ternární operátor}
V tomto algoritmu počítáme pouze s maticí délek a matici předchůdců vynecháváme. 
\lstset {language=C++}
\begin{lstlisting}
for ( int k = 0; k < n; ++k)
  for ( int i = 0; i < n; ++i) 
   for ( int j = 0; j < n; ++j) {
      D[i][j] = ( D[i][j] >  D[i][k] + D[k][j] ? D[i][k] + D[k][j] : D[i][j]);
   }
\end{lstlisting}

\begin{tabular}{|c|c|c|c|}
	\hline vlákna\textbackslash  uzlů  & 1000 & 1500 & 2000 \\ 
	\hline 1 & 1.53532 & .5.2044 & 12.8748 \\ 
	\hline 2 & 0.780031 & 2.61042 & 6.19777 \\ 
	\hline 4 & 0.385762 & 1.92644 & 3.08789 \\ 
	\hline 6 & 0.261746 & 1.28373 & 2.12849 \\ 
	\hline 8 & 0.208339 & 0.92291 & 1.81371 \\ 
	\hline 10 & 0.178032 & 0.773648 & 1.50141 \\ 
	\hline 12 & 0.152006 & 0.620251 & 1.35568 \\ 
	\hline 24 & 0.2881 & 0.57697 & 1.19574 \\ 
	\hline 
\end{tabular} 


\begin{tabular}{|c|c|c|c|}
	\hline vlákna\textbackslash  uzlů  & 1000 & 1500 & 2000 \\ 
	\hline 61 & 0.758129 & 1.67294 & 3.90219 \\ 
	\hline 122 & 0.622142 & 1.62244 & 3.20099 \\ 
	\hline 183 & 0.632614 & 1.51672 & 3.06845 \\ 
	\hline 244 & 0.669848 & 1.41912 & 2.76457 \\ 
	\hline 
\end{tabular} 

\subsubsection{min(a,b) metoda}
Nejaky text.
\lstset {language=C++}
\begin{lstlisting}
for ( int k = 0; k < n; ++k)
  for ( int i = 0; i < n; ++i) 
   for ( int j = 0; j < n; ++j) {
     D[i][j] = min( D[i][k] + D[k][j] ,  D[i][j] );
}
\end{lstlisting}

\begin{tabular}{|c|c|c|c|}
	\hline vlákna\textbackslash  uzlů & 1000 & 1500 & 2000 \\ 
	\hline 1 & 1.53201 & 5.2404 & 12.7781 \\ 
	\hline 2 & 0.777374 & 2.58762 & 6.18361 \\ 
	\hline 4 & 0.394649 & 1.31667 & 3.13383 \\ 
	\hline 6 & 0.269766 & 0.885247 & 2.15547 \\ 
	\hline 8 & 0.205032 & 0.778462 & 1.57456 \\ 
	\hline 10 & 0.175406 & 0.615799 & 1.3035 \\ 
	\hline 12 & 0.151809 & 0.622568 & 1.24625 \\ 
	\hline 24 & 0.245878 & 0.462049 & 1.11153 \\ 
	\hline 
\end{tabular} 

\begin{tabular}{|c|c|c|c|}
	\hline vlákna\textbackslash  uzlů & 1000 & 1500 & 2000 \\ 
	\hline 61 & 0.84409 & 2.05163 & 4.28279 \\ 
	\hline 122 & 0.673217 & 1.59254 & 3.06231 \\ 
	\hline 183 & 0.66181 & 1.5369 & 2.92522 \\ 
	\hline 244 & 0.668723 & 1.43921 & 2.78747 \\ 
	\hline 
\end{tabular} 

\subsubsection{1D pole}
Vyuziti 1d pole...

\lstset {language=C++}
\begin{lstlisting}
void NCG::FloydWarshall() {
for ( int k = 0; k < n; k++)
 for (int i = 0; i < n; i++)
   fw_inner(k, i);
}

void NCG::fw_inner(int k, int i) {
int d_ik = FWDistanceMatrix[i * nodes + k];
for (int j = 0; j < nodes; j++) {
 int d_kj = FWDistanceMatrix[k * nodes + j];
 int t = d_ik + d_kj;
 int ij = i * nodes + j;
 int d_ij = FWDistanceMatrix[ij];
 if (t < d_ij)
   FWDistanceMatrix[ij] = t;
}


}
}
\end{lstlisting}

\begin{tabular}{|c|c|c|c|}
	\hline vlákna\textbackslash  uzlů & 1000 & 1500 & 2000 \\ 
	\hline 1 & 2.30716 & 7.71098 & 18.4961 \\ 
	\hline 2 & 1.16078 & 3.86552 & 9.06954 \\ 
	\hline 4 & 0.588729 & 1.94404 & 4.55258 \\ 
	\hline 6 & 0.39564 & 1.3027 & 3.0499 \\ 
	\hline 8 & 0.303309 & 0.985183 & 2.37103 \\ 
	\hline 10 & 0.249257 & 0.793701 & 1.88205 \\ 
	\hline 12 & 0.237044 & 0.738506 & 1.63522 \\ 
	\hline 24 & 0.224236 & 0.69695 & 1.53367 \\ 
	\hline 
\end{tabular} 


icc:
LOOP BEGIN at fw1DArray.cpp(108,5)
 Peeled loop for vectorization, Multiversioned v1
remark \#15301: PEEL LOOP WAS VECTORIZED
LOOP END


\begin{tabular}{|c|c|c|c|}
	\hline vlákna\textbackslash  uzlů & 1000 & 1500 & 2000 \\ 
	\hline 61 & 0.418698 & 1.14185 & 2.35745 \\ 
	\hline 122 & 0.425456 & 0.961513 & 1.77961 \\ 
	\hline 183 & 0.448976 & 0.874202 & 1.68529 \\ 
	\hline 244 & 0.456448 & 0.820163 & 1.53792 \\ 
	\hline 
\end{tabular} 

Asi i poynamenat, ze kvuli vektorizaci pro mensi hodnoty a vice jader nam to bude kvuli tomu zpomalovat...


\subsubsection{Shrnutí}
Srovnání všechn hodnot pro g++ pro pocet uzlu 2000.

\begin{tabular}{|c|c|c|c|c|}
	\hline vlákna\textbackslash algoritmus & classic & ternarni operator & min(a,b) & 1D pole \\ 
	\hline 1 & 9.18048 & 12.8748 & 12.7781 & 18.4961 \\ 
	\hline 2 & 4.34206 & 6.19777 & 6.18361 & 9.06954 \\ 
	\hline 4 & 2.17533 & 3.08789 & 3.13383 & 4.55258 \\ 
	\hline 6 & 1.49812 & 2.12849 & 2.15547 & 3.0499 \\ 
	\hline 8 & 1.11342 & 1.81371 & 1.57456 & 2.37103 \\ 
	\hline 10 & 0.886158 & 1.50141 & 1.3035 & 1.88205 \\ 
	\hline 12 & 0.932699 & 1.35568 & 1.24625 & 1.63522 \\ 
	\hline 24 & 0.76749 & 1.19574 & 1.11153 & 1.53367 \\ 
	\hline 
\end{tabular} 

icc pro 2000
\begin{tabular}{|c|c|c|c|c|}
	\hline  & classic & ternarni operator & min(a,b) & 1D pole \\ 
	\hline 61 & 2.49671 & 3.90219 & 4.28279 & 2.35745 \\ 
	\hline 122 & 1.9717 & 3.20099 & 3.06231 & 1.77961 \\ 
	\hline 183 & 1.90224 & 3.06845 & 2.92522 & 1.68529 \\ 
	\hline 244 & 1.70336 & 2.76457 & 2.78747 & 1.53792 \\ 
	\hline 
\end{tabular} 

\subsubsection{Dodatek k vektorizaci}
mozna zminit ze existuji metody blocked and tiled, a ze jsou mozne dohledat treba tu:

% http://www.cs.virginia.edu/~pact2006/program/pact2006/pact139_han4.pdf
a dokonce nepochopitelny a desne slozity kod je dostupny na: 
% http://www.spiral.net/software/apsp.html

\end{document}